\subsection{ Report the value of the LQR gain for the proposed Q1 and Q2, as well as the evolution of the singular values of S obtained while solving the Riccati equation}
The proposed $Q_1$ and $Q_2$ are :
\begin{equation}
    Q_1 = 
    \left[ {\begin{array}{ccccc}
        1e-5 &0  &0   &0   &0     \\
        0    &50 &0   &0   &0     \\
        0    &0  &0.5 &0   &0     \\
        0    &0  &0   &0.5 &0     \\
        0    &0  &0   &0   &0.5   \\
    \end{array} } \right]    
    ,\quad
    Q_2 =
    \left[ {\begin{array}{cc}
        1 &0\\
        0 &2e-5\\
    \end{array} } \right]
\end{equation}

The value of the LQR gain obtained by solving the Riccati equation are :
\begin{equation}
    K_{LQR} = 
    \left[ {\begin{array}{ccccc}
         1.5641e-4 &0       &0      &0.2235 &0      \\
         0         &199.056 &722.53 &0      &19.474 \\
    \end{array}}\right]
\end{equation}

While the evolution of the singular values of $S$ obtained while while solving the Riccati equation is displayed below :
\begin{figure}[H]
    \centering
    \includegraphics[width = 0.7\linewidth]{Latex report/image/ex2/svds.png}
    \caption{Evolution of singular values of S while solving the Riccati equation}
    \label{fig:svds}
\end{figure}




\subsection{Report the explicit value of the observation close loop poles, and the observer gain resulting
from it.}
The poles of the observer were designed to be 99.9\% of the poles of the closed loop dynamics and the following values have been found (3 significant digits) :

\begin{equation}
    \text{Observer poles :}
    \left[\begin{array}{c}
         0.999\\
         0.987\\
         0.0154\\
         0.985 + 0.0137i\\
         0.985 - 0.0137i\\
    \end{array}
    \right]
\end{equation}

Then by placing the poles of the observer, one may find the following observer gain :

\begin{equation}
    L = 
    \left[ {\begin{array}{ccccc}
        0.9845 &0  &0      &0.01   &0         \\
        0      &50 &0.0299 &0      &0         \\
        0      &0  &0.0083 &0      &0.0008    \\
        0      &0  &0      &0.0032 &0         \\
        0      &0  &0      &0      &-0.0478   \\
    \end{array} } \right] 
\end{equation}

\subsection{Provide a screen shot of your final Simulink scheme and the submodules you had to complete.}
\begin{figure}[H]
    \centering
    \includegraphics[width = 0.8\linewidth]{Latex report/image/ex2Simulink.png}
    \caption{Simulink diagram of exercise 2}
    \label{fig:ex2Simulink}
\end{figure}


\subsection{Report the simulation results for the proposed values.}


\subsection{The proposed value of Q1 intends to assign very little importance to the error deriving from x1. Could you explain why doing this might make sense?}


\subsection{Propose a different pair Q1 and Q2 such that heading deviation error is highly penalized compared to the other states and report simulation results where the impact of doing so can be clearly observed.}


\subsection{Is the proposed placement for the observation close loop poles appropriate? why? If not,propose a new set of poles to improve the observation performance.}