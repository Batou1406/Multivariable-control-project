\subsection{ Report the value of the LQR gain for the proposed Q1 and Q2, as well as the evolution of the singular values of S obtained while solving the Riccati equation}
The proposed $Q_1$ and $Q_2$ are :
\begin{equation}
    Q_1 = 
    \left[ {\begin{array}{ccccc}
        1e-5 &0  &0   &0   &0     \\
        0    &50 &0   &0   &0     \\
        0    &0  &0.5 &0   &0     \\
        0    &0  &0   &0.5 &0     \\
        0    &0  &0   &0   &0.5   \\
    \end{array} } \right]    
    ,\quad
    Q_2 =
    \left[ {\begin{array}{cc}
        1 &0\\
        0 &2e-5\\
    \end{array} } \right]
\end{equation}

The value of the LQR gain obtained by solving the Riccati equation are :
\begin{equation}
    K_{LQR} = 
    \left[ {\begin{array}{ccccc}
         1.5641e-4 &0       &0      &0.2235 &0      \\
         0         &199.056 &722.53 &0      &19.474 \\
    \end{array}}\right]
\end{equation}


\subsection{Report the explicit value of the observation close loop poles, and the observer gain resulting
from it.}


\subsection{Provide a screen shot of your final Simulink scheme and the submodules you had to complete.}


\subsection{Report the simulation results for the proposed values.}


\subsection{The proposed value of Q1 intends to assign very little importance to the error deriving from x1. Could you explain why doing this might make sense?}


\subsection{Propose a different pair Q1 and Q2 such that heading deviation error is highly penalized compared to the other states and report simulation results where the impact of doing so can be clearly observed.}


\subsection{Is the proposed placement for the observation close loop poles appropriate? why? If not,propose a new set of poles to improve the observation performance.}